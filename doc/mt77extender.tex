
Para soportar otros formatos de documentos, en el momento
de este escrito debe:
\begin{enumerate}
\item Incluir un caso de detecci�n del formato en la funci�n 
\verb|determinaFormato| del archivo \verb|Operaciones.cpp|.
\item Incluir en el proceso de compilaci�n un nuevo archivo donde est�
la funci�n de conversi�n (i.e en el archivo \verb|Makefile| ver ejemplo
\verb|leeXML.cpp|), esta funci�n t�picamente recibir� el nombre
del archivo, el n�mero de documento al que corresponde y el �rbol 
\verb|NodoTrieS|
donde deben a�adirse las palabras.  Lo que debe hacer es abrir el archivo,
leer las palabras indexables y su posici�n y agregarlas al �rbol que recibe.
Para a�adir las palabras una a una puede emplear la funci�n
\verb|inserta| (ver ejemplo en funci�n \verb|leeXML|) o bien puede
convertir todo el documento a texto plano y emplear la funci�n 
\verb|leeTexto| (ver ejemplo en funci�n \verb|leeODT|).
\item Incluir un caso de conversi�n en la funci�n
\verb|indexa| de \verb|Operaciones.cpp| que llame la funci�n del punto anterior
cada vez que se lea un archivo en el nuevo formato.
\end{enumerate}


