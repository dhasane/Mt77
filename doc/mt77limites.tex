\item Sistema operativo: Con FFS2 sistemas de archivos y particiones
m�ximo de 2TB i.e $2^{41}$ bytes.
\item Formato: M�xima cantidad de documentos que puede indexar: $2^{31}$ 
M�ximo tama�o de un documento: $2^{31}$ = 2GB. M�ximo tama�o de un
�ndice: $2^{31}$ = 2GB, lo cual a su vez limita el m�ximo tama�o de un documento
indexable a $2^{31}\over 2$ = 1GB.  El m�todo de compresi�n empleado mejora al 
indexar documentos de menos de 64K.
\item Indexador de un documento: Opera en RAM, en experimentos ha logrado indexar documentos
de hasta 60MB.   Aunque limitado, lo consideramos suficiente porque no es 
t�pico tener documentos de tales tama�os y el mezclador permite manejar
colecciones mucho mayores.
\item Mezclador: En RAM s�lo requiere conjunto de posiciones de un nodo,
las cuales seg�n experimentos con 1GB en RAM alcanzan para indexar
una misma palabra aproximadamente en 10 millones de posiciones. La
limitaci�n en su operaci�n la impone el espacio disponible en
disco.  Para mezclar 2 �ndices de tama�os $P$ y $Q$ requiere
espacio $P+Q$ temporales, es decir si todas las mezclas se hacen en la misma
partici�n $2P+2Q$.  O si la colecci�n de documentos de una
organizaci�n es de tama�o $T$, el �ndice ser� de tama�o $0.6 T$ y
la construcci�n del mismo requerir� un espacio similar es decir
$2.2 T$. i.e. m�ximo 9GB en una partici�n de 20GB.


