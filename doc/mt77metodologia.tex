\subsubsection{Estudio de Caso: Amberfish}

Amberfish permite indexar colecciones de documentos y hacer b�squedas
sobre los �ndices.  Su autor previamente hab�a desarrollado {\tt isearch}.

Desventajas: no soporta ISO-8859-1, al parecer s�lo ASCII de 7 bits.
(ver \cite{amberunicode}).

Ventajas: Permite b�squedas booleanas, opera sobre archivos XML de forma 
gen�rica.

%Requiere menos tiempo en compilaci�n y en construcci�n de �ndices, genera �ndices m�s peque�os. Los tiempos de b�squeda que obtuvimos fueron mejores desde el primer prototipo (ver \cite{conclusionesbuscar}).



\subsubsection{Estudio de Caso: Lucene}

Es un buscador muy popular con a�os de desarrollo
en el marco del proyecto Apache.  Hay bastantes programas que lo incluyen
y otros que son compatibles con sus �ndices binarios.  En cuanto
a motores de b�squeda, Lucene es posiblemente el par�metro de comparaci�n 
est�ndar.

Ventajas: modularidad, dise�o maduro, tama�o de �ndices y velocidad de indexado.

Desventajas: escrito en Java que no ha sido tan portable en la pr�ctica.

%Lucene requiere menos tiempo de compilaci�n y en la construcci�n de �ndices y aunque los �ndices que genera son muy peque�os, los tiempos de b�squeda que obtuvimos fueron mejores desde el primer prototipo (ver \cite{conclusionesbuscar}).


Las fuentes de  Lucene constan de varios paquetes y clases bien divididos
e ideados.  En \cite{nagi} se describe una extensi�n a Lucene, y se 
describen brevemente las 2 operaciones que Lucene realiza: indexar y buscar.  
El proceso de {\em indexado} se hace sobre
documentos puestos a disposici�n por el {\em recolector
de datos,} que el  {\em reconocedor} convierte a textos planos. En la 
fase de {\em an�lisis} los datos se dividen de acuerdo a delimitadores
predefinidos y se realizan algunas operaciones sobre estos --por ejemplo
convertir a min�sculas, retirar algunas palabras, convertir a
forma ra�z.   El proceso de {\em b�squeda} primero {\em reconoce} la
consulta del usuario  para extraer sus unidades y operandos, estos pasan
al mismo {\em analizador} del  indexado, despu�s el �ndice se recorre en
busca de coincidencias y se retorna una lista de hallazgos ordenados. 
El {\em procesador de consultas difuso} define los criterios de coincidencia
y el puntaje de cada hallazgo.


